% use pdflatex
\documentclass[10pt]{article}
\oddsidemargin = 0.2in
\topmargin = -0.5in
\textwidth 6in
\textheight 8.5in

\usepackage{graphicx,bm,hyperref,amssymb,amsmath,amsthm}
\usepackage{algorithmic,xcolor}

\usepackage{showlabels}

% -------------------------------------- macros --------------------------
% general ...
\newcommand{\bi}{\begin{itemize}}
\newcommand{\ei}{\end{itemize}}
\newcommand{\ben}{\begin{enumerate}}
\newcommand{\een}{\end{enumerate}}
\newcommand{\be}{\begin{equation}}
\newcommand{\ee}{\end{equation}}
\newcommand{\bea}{\begin{eqnarray}} 
\newcommand{\eea}{\end{eqnarray}}
\newcommand{\ba}{\begin{align}} 
\newcommand{\ea}{\end{align}}
\newcommand{\bse}{\begin{subequations}} 
\newcommand{\ese}{\end{subequations}}
\newcommand{\bc}{\begin{center}}
\newcommand{\ec}{\end{center}}
\newcommand{\bfi}{\begin{figure}}
\newcommand{\efi}{\end{figure}}
\newcommand{\ca}[2]{\caption{#1 \label{#2}}}
\newcommand{\ig}[2]{\includegraphics[#1]{#2}}
\newcommand{\bmp}[1]{\begin{minipage}{#1}}
\newcommand{\emp}{\end{minipage}}
\newcommand{\pig}[2]{\bmp{#1}\includegraphics[width=#1]{#2}\emp} % mp-fig, nogap
\newcommand{\bp}{\begin{proof}}
\newcommand{\ep}{\end{proof}}
\newcommand{\ie}{{\it i.e.\ }}
\newcommand{\eg}{{\it e.g.\ }}
\newcommand{\etal}{{\it et al.\ }}
\newcommand{\pd}[2]{\frac{\partial #1}{\partial #2}}
\newcommand{\pdc}[3]{\left. \frac{\partial #1}{\partial #2}\right|_{#3}}
\newcommand{\infint}{\int_{-\infty}^{\infty} \!\!}      % infinite integral
\newcommand{\tbox}[1]{{\mbox{\tiny #1}}}
\newcommand{\mbf}[1]{{\mathbf #1}}
\newcommand{\half}{\mbox{\small $\frac{1}{2}$}}
\newcommand{\C}{\mathbb{C}}
\newcommand{\N}{\mathbb{N}}
\newcommand{\R}{\mathbb{R}}
\newcommand{\Z}{\mathbb{Z}}
\newcommand{\RR}{\mathbb{R}^2}
\newcommand{\ve}[4]{\left[\begin{array}{r}#1\\#2\\#3\\#4\end{array}\right]}  % 4-col-vec
\newcommand{\vt}[2]{\left[\begin{array}{r}#1\\#2\end{array}\right]} % 2-col-vec
\newcommand{\mt}[4]{\left[\begin{array}{ll}#1&#2\\#3&#4\end{array}\right]} % 2x2, rowwise ordering
\newcommand{\bigO}{{\mathcal O}}
\newcommand{\qqquad}{\qquad\qquad}
\newcommand{\qqqquad}{\qqquad\qqquad}
\DeclareMathOperator{\Span}{Span}
\DeclareMathOperator{\im}{Im}
\DeclareMathOperator{\re}{Re}
\DeclareMathOperator{\diag}{diag}
\newtheorem{thm}{Theorem}
\newtheorem{cnj}[thm]{Conjecture}
\newtheorem{lem}[thm]{Lemma}
\newtheorem{cor}[thm]{Corollary}
\newtheorem{pro}[thm]{Proposition}
\newtheorem{rmk}[thm]{Remark}
\newtheorem{dfn}[thm]{Definition}
\newtheorem{cno}[thm]{Code node}
% this work...
\newcommand{\x}{\mathbf{x}}
\newcommand{\y}{\mathbf{y}}
\newcommand{\vv}{\mathbf{v}}
\newcommand{\qq}{\mathbf{q}}
\newcommand{\n}{\mathbf{n}}
\newcommand{\f}{\mathbf{f}}
\newcommand{\bal}{\bm{\alpha}}
\newcommand{\bga}{\bm{\gamma}}
\newcommand{\eps}{\varepsilon}
\newcommand{\emach}{\eps_\tbox{mach}}
\newcommand{\E}{\R^3\backslash\overline{\Omega}}    % ext domain
\newcommand{\pO}{\partial\Omega}
\newcommand{\ok}{^{(k)}}
\newcommand{\okp}{^{(k')}}
\newcommand{\SR}{{\cal S}}           % MFS (aka SLP) rep

\begin{document}
\title{Method of fundamental solutions for large-scale 3D elastance problems}
\author{Alex H. Barnett}
\date{\today}
\maketitle
\begin{abstract}
  The MFS is effective for the 3D Laplace Dirichlet BVP in the
  exterior of a collection of simple objects, for example, spheres.
  One-body dense direct preconditioning (following
  Liu--Barnett) results in a
  well-conditioned square linear system that may be solved iteratively,
  with an FMM-accelerated matrix-vector apply,
  giving a linear-scaling scheme that may solve thousands of spheres
  in an hour on a workstation.
  Here we present a scheme for the {\em elastance} BVP in the same
  geometries, including its one-body preconditioning.
  This solves the problem of conductors with known net charges
  but unknown constant potentials.
  As far as we are aware, the MFS has not been used for such problems
  in prior literature.
  We do not discuss issues of MFS representations for
  close-to-touching spheres, and thus stick to unit spheres with surface
  separations $\delta \ge 0.1$.
  The idea is expected to generalize without fuss to the Stokes equations.
\end{abstract}

\section{The Dirichlet, capacitance and elastance BVPs}

Let $\Omega_k$, $k=1,\dots,K$ be a collection of smooth bounded
disjoint objects in $\R^3$,
each of which has boundary $\pO_k$,
and let $\Omega := \bigcup_k \Omega_k$ denote the collection,
and $\pO$ denote the union of all boundaries.
Given boundary voltage data $f$, the  Dirichlet BVP is to find $u$
such that
\bea
\Delta u &=& 0 \qquad \mbox{ in } \E
\label{pde}
\\
u &=& f\qquad\mbox{ on } \pO.
\label{dbc}
\eea
The special case where the data in \eqref{dbc} takes the form
\be
u = v_k\qquad\mbox{ on } \pO_k
\label{vk}
\ee
where $v_k\in\R$ is a given constant voltage on the $k$th boundary,
is called the {\em capacitance} problem,
and the resulting net charges
\be
q_k := -\int_{\pO_k} u_n ds
\label{qk}
\ee
are of interest,
where $u_n:=\partial u/\partial n = \n\cdot \nabla u$,
and $\n$ is the outwards unit normal on the boundary.

The solution defines a linear map from the input voltage vector
$\vv := \{v_k\}_{k=1}^K$ to the output charge vector $\qq := \{q_k\}_{k=1}^K$,
ie, a positive semidefinite capacitance matrix $C\in \R^{K\times K}$ acting as
\be
\qq = C \vv.
\ee
The full matrix $C$ could be extracted by solving $K$ of the above BVPs.

The elastance problem is the inverse of the capacitance problem.
Namely, given $q_k$, $k=1,\dots,K$, one is to find
the solution potential $u$ obeying the PDE \eqref{pde},
the given net charges \eqref{qk}, and
constant on each boundary as in \eqref{vk}.
The voltages $v_k$ are unknown and must be solved as part of the problem.
This is equivalent to applying the matrix inverse
$\vv = C^{-1} \qq$ where $C^{-1}$ is the elastance matrix.

\begin{rmk}
A useful numerical test is to start with a known $\vv$ vector,
solve the capacitance problem to get $\qq$, then solve the elastance
problem with this as input to get an output vector $\vv'$,
finally reporting $\|\vv'-\vv\|_\infty/\|\vv\|_\infty$ as an error metric.
\end{rmk}

\begin{rmk}
  The $\R^2$ case is more complicated due to additional constraints
  on total charge and a more subtle asymptotic form as $|\x|\to\infty$.
  See \cite{mobility}.
  Furthermore, in $\R^2$ there are deficiencies in the charge
  MFS representations on proxy surfaces with unit logarithmic capacity
  \cite{qfs} that require rank-1 corrections that would interact with the
  rank-1 corrections below. For this reason we stick to $\R^3$.
  \end{rmk}

\begin{rmk}
  The Stokes analog of capacitance is the resistance problem for rigid bodies,
  and the analog of elastance is mobility for rigid bodies. See \cite{mobility}.
  \end{rmk}

\section{MFS for the Dirichlet BVP}

Recall the Laplace fundamental solution,
\be
\Phi(\x,\y) = \Phi(\x-\y) = \frac{1}{4\pi \|\x-\y\|},
\ee
where $\x\in\R^3$ is a target point and $\y\in\R^3$ a source point.
This obeys $-\Delta \Phi(\cdot,\y) = \delta_\y$ in the distributional
sense.
Let $\y\ok_j$, $j=1,\dots,N$ be proxy (source) points inside $\Omega_k$
and $\x\ok_i$, $i=1,\dots,M$ be collocation (surface) points on $\pO_k$.
Assume that $N$ and $M$ are independent of the body index $k$ for simplicity.
Then the block of the MFS matrix $S^{(kk')}$
from sources in body $k'$ to targets on body $k$
has entries
\be
S^{(kk')}_{ij} = \Phi(\x\ok_i,\y\okp_j), \quad i=1,\dots,M, \quad j=1,\dots,N.
\ee
The symbol $S$ reminds one that this matrix is a crude Nystr\"om
discretization of the 1st-kind layer operator from the $k$'th proxy surface
to the $k$th boundary.
The diagonal (self-interaction) matrix $S^{(kk)}$
is exponentially ill-conditioned as $N$ grows, with $M$ somewhat
larger than $N$. A typical choice is $M\approx 1.2 N$.

\begin{rmk}
  For spheres we recommend spherical design points for surface and proxy
  points \cite{sphdesign},
which are available for $N$ up to around 16000 at
\url{https://web.maths.unsw.edu.au/~rsw/Sphere/EffSphDes/}
Proxy points are scaled to a sphere of radius $R_p \approx 0.7$
when targeting several digit accuracy in the case of $\Omega_0$ the unit sphere.
Here $N$ in the range $500$ to $2000$ gives
$\kappa(S^{(kk)})$ in the range $10^5$ to $10^9$.
\end{rmk}

In the case $K=1$, the MFS then solves in the least-squares sense the
formally overdetermined $M\times N$ system with matrix $S^{(11)} = S$,
\be
S \bal = \f
\label{1sys}
\ee
where $\f:=\{f(\x_i)\}_{i=1}^M$ is the data at collocation points
and $\bal:=\{\alpha_j\}_{j=1}^N$ are the proxy strengths (coefficients).
For $K=2$ we have
\be
\mt{S^{(11)}}{S^{(12)}}{S^{(21)}}{S^{(22)}} \vt{\bal^{(1)}}{\bal^{(2)}} = \vt{\f^{(1)}}{\f^{(2)}}.
\label{2sys}
\ee
We do not write the generalization to $K>2$ since it is obvious.
In the case where all proxy and collocation point sets are
translates of those for a single body,
all self-interaction matrices are the same,
$S^{(kk)} = S$ for all $k=1,\dots,K$.

Having solved \eqref{2sys} for the stacked solution vector
$\bal:=\{\bal\ok\}_{k=1}^K$, the representation for the
solution is
\be
u(\x) = \sum_{k=1}^K \sum_{j=1}^N \alpha\ok_j \Phi(\x,\y\ok_j),
\qquad \x\in\E,
\label{rep}
\ee
which we abbreviate by $u = \SR\bal$ by analogy with layer potentials.

The net charges $q_k$ may be extracted either by evaluation of
\eqref{qk} (which requires an accurate quadrature weights
for the set of collocation points, and evaluations of $\nabla \Phi$),
or more conveniently via
\be
q_k = \sum_{j=1}^N \alpha\ok_j, \qquad k=1,\dots,K,
\label{qkgauss}
\ee
which follows by Gauss' law for $\Phi$.

The above method is not very practical, since it requires a
dense direct least-squares solution.


\section{One-body preconditioning for larger-scale Dirichlet problems}
\label{s:dirprec}

The full least-squares
system \eqref{2sys} is large, dense, and ill-conditioned,
thus not amenable to iterative solution.
Here we recall how to precondition it to allow an accelerated iterative
solution.
Let all self-interaction matrices be the same $S$, for simplicity.
Let $S = U \Sigma V^T$ be the SVD,
with $\Sigma$ the diagonal matrix with entries the singular values $\sigma_j$,
then
the solution operator to the the system \eqref{1sys}
is the pseudoinverse
$$
\bal = S^+ \f := V \Sigma^+ (U^T \f),
$$
where $\Sigma^+$ is the diagonal matrix with entries $1/\sigma_j$.
(Sometimes regularization is used for small $\sigma_j$.)
Note that $S^+$ cannot be applied stably by forming it as a matrix;
the two-step application above is needed, and implied whenever we
write $S^+$ in what follows.
We apply right-preconditioning by $S^+$, as in \cite{acper}.
\footnote{This idea was suggested to us by Arvind K. Saibaba in 2014.}
Let the new unknowns (which represent surface values)
be given by the vectors $\bga\ok = S \bal\ok$,
for $k=1,\dots,K$.
Noting that $S S^+ = I$, in the case where $M\ge N$ and $S$ has
numerical rank $N$, then
\eqref{2sys} becomes
\be
\mt{I}{S^{(12)}S^+}{S^{(21)}S^+}{I} \vt{\bga^{(1)}}{\bga^{(2)}} = \vt{\f^{(1)}}{\f^{(2)}}.
\label{2sysprec}
\ee
One solves this iteratively for the stacked surface value vector
$\bga := \{\bga\ok\}_{k=1}^K$,
then recovers $\bal\ok = S^+ \bga\ok$ for each $k$.
For large-scale problems ($KM \gg 10^4$) in the iterative solve the
square mat-vec is applied in four stages:
\ben
\item apply $S^+$ to each block of the vector to get strengths at the proxy points,
\item use the $KN$ source to $KM$ target FMM to apply the bare
MFS system matrix appearing in \eqref{2sys},
\item in each block subtract $S$ times the block from the answer
  to kill the diagonal blocks, then
\item
  add the original vector to
account for the identity in \eqref{2sysprec}.
\een

\begin{cno}
In the attached Julia codes, one may test the above for spheres using
{\tt multi3delast.jl} with {\tt elast=false}.
No FMM acceleration is included yet.
\end{cno}


\section{Elastance formulation}

% intuitive
We propose an MFS elastance formulation
inspired by ``completion'' or ``compound'' flows in the Stokes literature
(Power--Miranda, Pozrikidis, etc, following Mikhlin).
Recall that in that setting the double-layer is unable to apply a net force
or torque, so must be augmented by other types of source.
In constrast, for us the proxy charges {\em can} apply net charge,
so we emulate a zero-net-charge source by projection and use
the proxy points themselves as the completion sources.

% notation
We use the notation $\mbf{1}$, sometimes with a subscript to denote the dimension, for the vector with all entries $1$.
Let $L = \frac{1}{N}\mbf{1}\mbf{1}^T$ be the $N\times N$ matrix
with all entries $1/N$; it is the orthogonal projector
onto the constant vectors in $\R^N$,
while $I-L$ is the orthogonal projector onto the complement space.
We also need the rectangular $M\times N$ matrix
$L_r$ with all entries $1/N$.

% define al_0
Recalling the representation notation $\SR$ from above,
we set up a ``completion potential'' $\SR\bal_0$
where $\bal_0\in\R^{KN}$ is a strength vector,
which imparts the desired net charge $q_k$ to the $k$th body.
A simple choice of $\bal_0$ is each block constant,
\be
\bal_0\ok = \frac{q_k}{N}\mbf{1}_N, \qquad k=1,\dots,K.
\label{al0}
\ee

% one body
We explain first the case $K=1$ for simplicity.
Given an unknown vector $\bal\in\R^{N}$,
our proposed representation is
\be
u = \SR (I-L)\bal + \SR\bal_0 \qqqquad \mbox{($K=1$ case)},
\label{erep1}
\ee
where we have projected out the nonzero mean part of $\bal$ so that the
first term cannot change the net body charge, by analogy
with a double-layer potential.
Inserting \eqref{erep1} into the boundary condition \eqref{vk}
(being $u|_{\pO_1} = v_1$) at collocation nodes gives
\be
S(I-L)\bal + S\bal_0 = \mbf{V}
\label{esys1}
\ee
where $S$ is the MFS matrix, and $\mbf{V} = v_1\mbf{1}_M$ the constant
vector with unknown constant (the notation $\mbf{V}$ echoes
that for a rigid body motion in the Stokes literature).
The system is closed by choosing a representation for $V$ in terms of
$\bal$. The subspace Span$\{\mbf{1}_N\}$ is available for this, it
having no effect on the first term $S(I-L)\bal$.
Hence we make an ansatz
\[
\mbf{V} = -L_r \bal,
\]
analogous to the Stokes mobility case
(see, eg, \cite[Sec.~4.2]{csbq}),
and substituting this into \eqref{esys1} gives the linear system
\[
\bigl[ S(I-L) + L_r \bigr] \bal = -S\bal_0.
\]
It is easy to check that if $\bal$ solves this system, then
the boundary condition is satisfied at collocation nodes
for some constant $v_1 = -\frac{1}{N}\mbf{1}^T\bal$.
The net charge is $q_1$ by construction, and the representation \eqref{erep1}
satisfies the PDE, so that it is the unique solution to the elastance
problem.

% general
We now present the case for a general number of bodies $K\ge 1$.
The representation is
\be
u = \sum_{k=1}^K \SR\ok \bigl[ (I-L)\bal\ok + \bal_0\ok \bigr],
\label{erepK}
\ee
where $\SR\ok$ is the MFS charge representation from body $k$;
more precisely
$(\SR\ok \bm{\beta})(\x) := \sum_{j=1}^N \beta_j \Phi(\x,\y\ok_j)$
for any vector $\bm{\beta}:=\{\beta_j\}_{j=1}^N$.
The algorithm is summarized by the main result.
\begin{pro} Let $\bal\in\R^{KN}$ solve the linear system with block rows
  \be
     [S(I-L) + L_r] \bal\ok + \sum_{k'\neq k} S^{(kk')}(I-L)\bal\okp = -\mbf{u}\ok_0
     ,\qquad k=1,\dots,K,
  \label{esysK}
  \ee
  where, recalling $\bal_0$ defined by \eqref{al0}, the right-hand side vector has block entries
  \be
  \mbf{u}\ok_0 = \sum_{k'=1}^K S^{(kk')} \bal\okp_0, \qquad k=1,\dots,K.
  \label{erhsK}
  \ee
  Then the potential $u$ given by the representation \eqref{erepK} is the unique solution
  to the discretized elastance BVP, meaning that
  it is harmonic in $\E$, constant
  on the collocation nodes for each body, and has the
  desired net charges \eqref{qkgauss}.
  The constant voltages may be read off as the negative mean strengths
  \be
  v_k = -\frac{1}{N}\sum_{j=1}^N \alpha\ok_j, \qquad k=1,\dots,K.
  \label{vkal}
  \ee
\end{pro}
%block-constant vector $V = [v_1 \mbf{1}_M; \dots ; v_K \mbf{1}_M]$ encodes the unknown voltages $v_k$.
\begin{proof}
  Replacing $k$ by $k'$ in the representation \eqref{erepK},
  then evaluating it on the collocation nodes of the $k$th body,
  gives the vector of potentials
  \[
  \{u(\x\ok_i)\}_{i=1}^M =
  \sum_{k'=1}^K S^{(kk')}\bigl[  (I-L)\bal\okp + \bal\okp_0 \bigr].
  \]
  Subtracting \eqref{esysK} leaves $\{u(\x\ok_i)\}_{i=1}^M = -L_r \bal\ok$,
  which shows that the potential is constant on the $k$th body,
  and that constant is \eqref{vkal}.
  $u$ is harmonic since it is a sum of fundamentation solutions,
  and has the correct charges by construction from \eqref{al0}.
\end{proof}

The least squares linear system \eqref{esysK} is not very practical
since it requires a dense direct solution, taking $\bigO(K^3N^3)$ effort.
We improve this in the next subsection.


\begin{rmk}[Alternative approach to elastance]\label{r:saddle}
A naive approach to elastance might augment the MFS system
with extra unknowns (the $v_k$) and extra conditions
(the conditions \eqref{qkgauss} on the coefficients).
For $K=1$ bodies this leads to a saddle-point type
rectangular linear system
with block structure
$$
\mt{S}{-\mbf{1}_M}{\mbf{1}_N^T}{0} \vt{\bal}{v_1} = \vt{\mbf{0}}{q_1}.
$$
By elimination of $\bal$ the solution is $v_1 = q_1 / (\mbf{1}^T S^+ \mbf{1})$,
where we inverted the $1\times 1$ capacitance matrix $\mbf{1}^T S^+ \mbf{1}$.
The generalization to $K>1$ appears to need $K$ solves:
it basically involves filling (via Dirichlet BVP solves), then inverting, the $K\times K$ capacitance matrix, at a cost $\bigO(NK^2 + K^3)$.
We cannot see how to turn the augmented elastance system into a square system that
may be preconditioned and solved (once) iteratively using the FMM
at cost $\bigO(NK)$.
``Ones-matrix'' tricks of adding a rank-one to $S$ are unclear
because $S$ is usually numerically full rank.
\end{rmk}


\section{One-body preconditioning for elastance}

The elastance MFS linear system may be right-preconditioned by factorizing
the diagonal blocks, identically to the Dirichlet BVP case.
Letting $S_L := S(I-L) + L_r \in \R^{M\times N}$,
then in the $K=2$ case the result is
\be
\mt{I}{S^{(12)}(I-L)S_L^+}{S^{(21)}(I-L)S_L^+}{I} \vt{\bga^{(1)}}{\bga^{(2)}} = \vt{\mbf{u}^{(1)}}{\mbf{u}^{(2)}},
\label{2esysprec}
\ee
recalling the right-hand side definition \eqref{erhsK}.
The generalization to $K>2$ is obvious.
One first precomputes the two matrices that apply the pseudoinverse
$S^+_L$, using its SVD.
Then, one solves \eqref{2esysprec}
iteratively using GMRES for the stacked surface
value vector
$\bga := \{\bga\ok\}_{k=1}^K$.
An accelerated mat-vec would proceed as in Sec.~\ref{s:dirprec} except
with the application of $I-L$ to each vector block after Step 1.
One then recovers $\bal\ok = S^+ \bga\ok$ for each $k$,
finally reporting their averages $v_k$ via \eqref{vkal}.

\begin{cno}
  In the attached Julia codes, one may test the above for spheres using
{\tt multi3delast.jl} with {\tt elast=true}.
No FMM acceleration is yet included. Even filling a dense matrix
for the mat-vec, $KN$ up to about $20000$ are
possible in a few seconds on an 8-core CPU. See Fig.~\ref{f:elast}.
  \end{cno}

\bfi
\centering \ig{width=5in}{un_K20_d0.1.png}
\ca{Elastance calculation with $K=20$ unit spheres with minimum separation
  $\delta=0.1$, $N=969$ unknowns per sphere, 5 digits of accuracy in the maximum relative boundary condition error. Shown is $u_n$ (charge density)
  at $2KM= 44920$ test points. Solution time about 20 seconds.}{f:elast}
\efi


\section{Open questions}

\ben
\item The Stokes versions.
\item Remind reader of uniqueness and existence for elastance (Rachh).
\item Rigorous proof at continuous level of existence of MFS solution assuming
  continuation inside as a PDE solution. Same for elastance formulation---%
  how do we know the loss of constant functions doesn't cause any resonances
  or problems?
\item Is Rmk.~\ref{r:saddle} correct? Search for connection
  of augmented to completion systems.
\een


\bibliographystyle{abbrv}
\bibliography{refs}

\end{document}
