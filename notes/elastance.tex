% use pdflatex
\documentclass[10pt]{article}
\oddsidemargin = 0.2in
\topmargin = -0.5in
\textwidth 6in
\textheight 8.5in

\usepackage{graphicx,bm,hyperref,amssymb,amsmath,amsthm}
\usepackage{algorithmic,xcolor}

\usepackage{showlabels}

% -------------------------------------- macros --------------------------
% general ...
\newcommand{\bi}{\begin{itemize}}
\newcommand{\ei}{\end{itemize}}
\newcommand{\ben}{\begin{enumerate}}
\newcommand{\een}{\end{enumerate}}
\newcommand{\be}{\begin{equation}}
\newcommand{\ee}{\end{equation}}
\newcommand{\bea}{\begin{eqnarray}} 
\newcommand{\eea}{\end{eqnarray}}
\newcommand{\ba}{\begin{align}} 
\newcommand{\ea}{\end{align}}
\newcommand{\bse}{\begin{subequations}} 
\newcommand{\ese}{\end{subequations}}
\newcommand{\bc}{\begin{center}}
\newcommand{\ec}{\end{center}}
\newcommand{\bfi}{\begin{figure}}
\newcommand{\efi}{\end{figure}}
\newcommand{\ca}[2]{\caption{#1 \label{#2}}}
\newcommand{\ig}[2]{\includegraphics[#1]{#2}}
\newcommand{\bmp}[1]{\begin{minipage}{#1}}
\newcommand{\emp}{\end{minipage}}
\newcommand{\pig}[2]{\bmp{#1}\includegraphics[width=#1]{#2}\emp} % mp-fig, nogap
\newcommand{\bp}{\begin{proof}}
\newcommand{\ep}{\end{proof}}
\newcommand{\ie}{{\it i.e.\ }}
\newcommand{\eg}{{\it e.g.\ }}
\newcommand{\etal}{{\it et al.\ }}
\newcommand{\pd}[2]{\frac{\partial #1}{\partial #2}}
\newcommand{\pdc}[3]{\left. \frac{\partial #1}{\partial #2}\right|_{#3}}
\newcommand{\infint}{\int_{-\infty}^{\infty} \!\!}      % infinite integral
\newcommand{\tbox}[1]{{\mbox{\tiny #1}}}
\newcommand{\mbf}[1]{{\mathbf #1}}
\newcommand{\half}{\mbox{\small $\frac{1}{2}$}}
\newcommand{\C}{\mathbb{C}}
\newcommand{\N}{\mathbb{N}}
\newcommand{\R}{\mathbb{R}}
\newcommand{\Z}{\mathbb{Z}}
\newcommand{\RR}{\mathbb{R}^2}
\newcommand{\ve}[4]{\left[\begin{array}{r}#1\\#2\\#3\\#4\end{array}\right]}  % 4-col-vec
\newcommand{\vt}[2]{\left[\begin{array}{r}#1\\#2\end{array}\right]} % 2-col-vec
\newcommand{\mt}[4]{\left[\begin{array}{rr}#1&#2\\#3&#4\end{array}\right]} % 2x2, rowwise ordering
\newcommand{\bigO}{{\mathcal O}}
\newcommand{\qqquad}{\qquad\qquad}
\newcommand{\qqqquad}{\qqquad\qqquad}
\DeclareMathOperator{\Span}{Span}
\DeclareMathOperator{\im}{Im}
\DeclareMathOperator{\re}{Re}
\DeclareMathOperator{\diag}{diag}
\newtheorem{thm}{Theorem}
\newtheorem{cnj}[thm]{Conjecture}
\newtheorem{lem}[thm]{Lemma}
\newtheorem{cor}[thm]{Corollary}
\newtheorem{pro}[thm]{Proposition}
\newtheorem{rmk}[thm]{Remark}
\newtheorem{dfn}[thm]{Definition}
% this work...
\newcommand{\x}{\mathbf{x}}
\newcommand{\y}{\mathbf{y}}
\newcommand{\vv}{\mathbf{v}}
\newcommand{\qq}{\mathbf{q}}
\newcommand{\n}{\mathbf{n}}
\newcommand{\f}{\mathbf{f}}
\newcommand{\bal}{\bm{\alpha}}
\newcommand{\bga}{\bm{\gamma}}
\newcommand{\eps}{\varepsilon}
\newcommand{\emach}{\eps_\tbox{mach}}
\newcommand{\E}{\R^3\backslash\overline{\Omega}}    % ext domain
\newcommand{\pO}{\partial\Omega}
\newcommand{\ok}{^{(k)}}
\newcommand{\okp}{^{(k')}}


\begin{document}
\title{Method of fundamental solutions for large-scale 3D elastance problems}
\author{Alex H. Barnett}
\date{\today}
\maketitle
\begin{abstract}
  The MFS is effective for the 3D Laplace Dirichlet BVP in the
  exterior of a collection of simple objects, for example, spheres.
  One-body dense direct preconditioning (following
  Liu--Barnett) results in a
  well-conditioned square linear system that may be solved iteratively,
  with an FMM-accelerated matrix-vector apply,
  giving a linear-scaling scheme that may solve thousands of spheres
  in an hour on a workstation.
  Here we present a scheme for the {\em elastance} BVP in the same
  geometries, including its one-body preconditioning.
  This solves the problem of conductors with known net charges
  but unknown constant potentials.
  As far as we are aware, the MFS has not been used for such problems
  in prior literature.
  We do not discuss issues of MFS representations for
  close-to-touching spheres, and thus stick to unit spheres with surface
  separations $\delta \ge 0.1$.
  The idea is expected to generalize without fuss to the Stokes equations.
\end{abstract}

\section{The Dirichlet, capacitance and elastance BVPs}

Let $\Omega_k$, $k=1,\dots,K$ be a collection of smooth bounded
disjoint objects in $\R^3$,
each of which has boundary $\pO_k$,
and let $\Omega := \bigcup_k \Omega_k$ denote the collection,
and $\pO$ denote the union of all boundaries.
Given boundary voltage data $f$, the  Dirichlet BVP is to find $u$
such that
\bea
\Delta u &=& 0 \qquad \mbox{ in } \E
\label{pde}
\\
u &=& f\qquad\mbox{ on } \pO.
\label{dbc}
\eea
The special case where the data in \eqref{dbc} takes the form
\be
u = v_k\qquad\mbox{ on } \pO_k
\label{vk}
\ee
where $v_k\in\R$ is a given constant voltage on the $k$th boundary,
is called the {\em capacitance} problem,
and the resulting net charges
\be
q_k := -\int_{\pO_k} u_n ds
\label{qk}
\ee
are of interest,
where $u_n:=\partial u/\partial n = \n\cdot \nabla u$,
and $\n$ is the outwards unit normal on the boundary.

The solution defines a linear map from the input voltage vector
$\vv := \{v_k\}_{k=1}^K$ to the output charge vector $\qq := \{q_k\}_{k=1}^K$,
ie, a positive semidefinite capacitance matrix $C\in \R^{K\times K}$ acting as
\be
\qq = C \vv.
\ee
The full matrix $C$ could be extracted by solving $K$ of the above BVPs.

The elastance problem is the inverse of the capacitance problem.
Namely, given $q_k$, $k=1,\dots,K$, one is to find
the solution potential $u$ obeying the PDE \eqref{pde},
the given net charges \eqref{qk}, and
constant on each boundary as in \eqref{vk}.
The voltages $v_k$ are unknown and must be solved as part of the problem.
This is equivalent to applying the matrix inverse
$\vv = C^{-1} \qq$ where $C^{-1}$ is the elastance matrix.

\begin{rmk}
A useful numerical test is to start with a known $\vv$ vector,
solve the capacitance problem to get $\qq$, then solve the elastance
problem with this as input to get an output vector $\vv'$,
finally reporting the maximum norm of $\vv'-\vv$.
\end{rmk}

\begin{rmk}
  The $\R^2$ case is more complicated due to additional constraints
  on total charge and a more subtle asymptotic form as $|\x|\to\infty$.
  See \cite{mobility}.
  Furthermore, in $\R^2$ there are deficiencies in the charge
  MFS representations on proxy surfaces with unit logarithmic capacity
  \cite{qfs} that require rank-1 corrections that would interact with the
  rank-1 corrections below. For this reason we stick to $R^3$.
  \end{rmk}

\begin{rmk}
  The Stokes analog of capacitance is the resistance problem,
  and the analog of elastance is mobility. See \cite{mobility}.
  \end{rmk}

\section{MFS for the Dirichlet BVP}

Recall the Laplace fundamental solution,
\be
\Phi(\x,\y) = \Phi(\x-\y) = \frac{1}{4\pi \|\x-\y\|},
\ee
where $\x\in\R^3$ is a target point and $\y\in\R^3$ a source point.
This obeys $-\Delta \Phi(\cdot,\y) = \delta_\y$ in the distributional
sense.
Let $\y\ok_j$, $j=1,\dots,N$ be proxy (source) points inside $\Omega_k$
and $\x\ok_i$, $i=1,\dots,M$ be collocation (surface) points on $\pO_k$.
Assume that $N$ and $M$ are independent of the body index $k$ for simplicity.
Then the block of the MFS matrix $S^{(kk')}$
from sources in body $k'$ to targets on body $k$
has entries
\be
S^{(kk')}_{ij} = \Phi(\x\ok_i,\y\okp_j), \quad i=1,\dots,M, \quad j=1,\dots,N.
\ee
The symbol $S$ reminds one that this matrix is a crude Nystr\"om
discretization of the 1st-kind layer operator from the $k$'th proxy surface
to the $k$th boundary.
The diagonal (self-interaction) matrix $S^{(kk)}$
is exponentially ill-conditioned as $N$ grows, with $M$ somewhat
larger than $N$. A typical choice is $M\approx 1.2 N$.

\begin{rmk}
  For spheres we recommend spherical design points for surface and proxy
  points \cite{sphdesign},
which are available for $N$ up to around 16000 at
\url{https://web.maths.unsw.edu.au/~rsw/Sphere/EffSphDes/}
Proxy points are scaled to a sphere of radius $R_p \approx 0.7$
when targeting several digit accuracy in the case of $\Omega_0$ the unit sphere.
Here $N$ in the range $500$ to $2000$ gives
$\kappa(S^{(kk)})$ in the range $10^5$ to $10^9$.
\end{rmk}

In the case $K=1$, the MFS then solves in the least-squares sense the
formally overdetermined $M\times N$ system with matrix $S^{(11)} = S$,
\be
S \bal = \f
\label{1sys}
\ee
where $\f:=\{f(\x_i)\}_{i=1}^M$ is the data at collocation points
and $\bal:=\{\alpha_j\}_{j=1}^N$ are the proxy strengths.
For $K=2$ we have
\be
\mt{S^{(11)}}{S^{(12)}}{S^{(21)}}{S^{(22)}} \vt{\bal^{(1)}}{\bal^{(2)}} = \vt{\f^{(1)}}{\f^{(2)}}.
\label{2sys}
\ee
We do not write the generalization to $K>2$ since it is obvious.
In the case where all proxy and collocation point sets are
translates of those for a single body,
all self-interaction matrices are the same,
$S^{(kk)} = S$ for all $k=1,\dots,K$.

Having solved \eqref{2sys} for the stacked solution vector
$\bal:=\{\bal\ok\}_{k=1}^K$, the representation for the
solution is
\be
u(\x) = \sum_{k=1}^K \sum_{j=1}^N \alpha\ok_j \Phi(\x,\y\ok_j),
\label{rep}
\ee
which we abbreviate by $u = {\cal S}\bal$ by analogy with layer potentials.

The net charges $q_k$ may be extracted either by evaluation of
\eqref{qk} (which requires an accurate quadrature weights
for the set of collocation points, and evaluations of $\nabla \Phi$),
or more conveniently via
\be
q_k = \sum_{j=1}^N \alpha\ok_j, \qquad k=1,\dots,K,
\label{qkgauss}
\ee
which follows by Gauss' law for $\Phi$.


\section{One-body preconditioning for larger-scale problems}

The full system \eqref{2sys} is large, dense, and ill-conditioned,
thus not amenable to iterative solution.
Here we recall how to precondition it to allow an accelerated iterative
solution.
Let all self-interaction matrices be the same $S$, for simplicity.
Let $S = U \Sigma V^T$ be the SVD,
with $\Sigma$ the diagonal matrix with entries the singular values $\sigma_j$,
then
the solution operator to the the system \eqref{1sys}
is the pseudoinverse
$$
\bal = S^+ \f := V \Sigma^+ (U^T \f),
$$
where $\Sigma^+$ is the diagonal matrix with entries $1/\sigma_j$.
(Sometimes regularization is used for small $\sigma_j$.)
Note that $S^+$ cannot be applied stably by forming it as a matrix;
the two-step application above is needed, and implied whenever we
write $S^+$ in what follows.
We apply right-preconditioning by $S^+$, as in \cite{acper}.
\footnote{This idea was suggested to us by Arvind K. Saibaba in 2014.}
Let the new unknowns (which represent surface values)
be given by the vectors $\bga\ok = S \bal\ok$,
for $k=1,\dots,K$.
Noting that $S S^+ = I$, in the case where $M\ge N$ and $S$ has
numerical rank $N$, then
\eqref{2sys} becomes
\be
\mt{I}{S^{(12)}S^+}{S^{(21)}S^+}{I} \vt{\bga^{(1)}}{\bga^{(2)}} = \vt{\f^{(1)}}{\f^{(2)}}.
\label{2sysprec}
\ee
One solves this iteratively for the stacked surface value vector
$\bga := \{\bga\ok\}_{k=1}^K$,
then recovers $\bal\ok = S^+ \bga\ok$ for each $k$.
For large-scale problems ($KM \gg 10^4$) in the iterative solve the
square mat-vec is applied in four stages:
\ben
\item apply $S^+$ to each block of the vector to get strengths at the proxy points,
\item use the $KN$ source to $KM$ target FMM to apply the bare
MFS system matrix appearing in \eqref{2sys},
\item in each block subtract $S$ times the block from the answer
  to kill the diagonal blocks, then
\item
  add the original vector to
account for the identity in \eqref{2sysprec}.
\een

In the attached Julia codes, one may test the above using
{\tt multi3delast.jl} with {\tt elast = false}.
No FMM acceleration is included yet.


\section{Elastance formulation and its one-body preconditioning}

Let $\mbf{1}$ denote the vector with all entries 1.
Let $L = \frac{1}{N}\mbf{1}\mbf{1}^T$ be the $N\times N$ matrix
with all entries equal to $1/N$; it is the orthogonal projector
onto the constant vectors in $\R^N$,
while $I-L$ is the orthogonal projector onto the complement space.

Let 



\bibliographystyle{abbrv}
\bibliography{refs}

\end{document}
